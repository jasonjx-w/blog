\documentclass{ctexart}
\usepackage{graphicx}
\usepackage[colorlinks,linkcolor=black,anchorcolor=black,citecolor=black]{hyperref}
\usepackage{listings}


\begin{document}


%本章主要研究各种排序算法.
\section{排序}
不同排序算法根据时间复杂度可以分为:1)简单的排序算法($O(n^2)$);2)先进的排序算法($O(nlogn)$);3)基数排序($O(d\times n)$).

根据不同原理大体可以分为:1)插入排序;2)交换排序;3)选择排序;4)归并排序;5)计数排序.

存储方式有:数组,链表,数组+地址向量

\subsection{稳定与不稳定}
在比较过程中,当出现两个元素相等的情景.若出现元素的移动,则为不稳定排序,否则则为稳定排序算法.

\subsection{内部排序和外部排序}
待排序元素数量较小,在内存中实现的排序过程为内部排序;
当待排序元素数量较多,内存无法一次性容纳所有元素,在排序实现过程中需要访问外部存储的排序算法称为外部排序.

\subsection{几种排序算法介绍}
%插入排序算法
\subsubsection{插入排序}

	\subparagraph{思路}
	对于一个待排序序列,维护一个有序的子序列.不断的将待排元素插入到该有序子序列中,直至所有元素插入完毕,最终待排序列为有序序列.

	\subparagraph{代码}
	\begin{lstlisting}[language = C++, title=insert sort, frame=shadowbox, breaklines=true]
	template <typename T>
	void InsertSort(T* arr, int n)
	{
    f	or(int i=1;i<n;++i){ // for every item.
        	T tmp = arr[i];
        	int j = i;  // j must be i.
        	for(;j>0 && arr[j-1]>tmp;j--){ // move to right position.
            	arr[j] = arr[j-1];
        	}
        	arr[j] = tmp;
    	}
	}	
	\end{lstlisting}

	\subparagraph{复杂度分析}
	\subparagraph{变体}
	
	%希尔排序法
	\subsubsection{希尔排序}
	
	\subparagraph{思路}
	希尔排序法使用了\emph{增量}的概念.首先采用比如增量$h=n/2$,而$n$为序列总长度.每一轮只对位置为$a+h\times i$上的元素进行选择排序(其中$a$为第一个元素,显然$0\leq a < h$.而$i$为整数,显然$0 \leq a+h\times i< n$).而增量$h$逐渐减小,直至$h=1$.至此整个序列有序.
	\subparagraph{代码}

\subparagraph{代码}
\begin{lstlisting}[language = C++, title=insert sort, frame=shadowbox, breaklines=true]	
\end{lstlisting}

%选择排序法
\subsubsection{选择排序法}
\subparagraph{思路}
对于当前位置,应该从后继元素中,选择合适的元素(最大或最小)插入到当前位置.不断重复这个过程,直至整个序列有序.

\subparagraph{代码}
\begin{lstlisting}[language = C++, title=select sort, frame=shadowbox, breaklines=true]
template <typename T>
void SelectSort(T* arr, int n)
{
    for(int i=0;i<n;++i)  // for every item
    {
        int min_i=i;  // initial min_idx
        for(int j=i;j<n;++j)  // for every item behind current item.
        {
            if(arr[j]<arr[min_i]){  // find min_idx.
                min_i=j;
            }
        }
        T tmp = arr[i];  // exchange current item and minimum item.
        arr[i] = arr[min_i];
        arr[min_i]=tmp;
    }
}	
\end{lstlisting}
\subparagraph{复杂度分析}
\subparagraph{变体}



%\bibliographystyle{plain}
%\bibliography{refs.bib}
	
\end{document}