\documentclass{ctexart}
\usepackage{graphicx}
\usepackage[colorlinks,linkcolor=black,anchorcolor=black,citecolor=black]{hyperref}
\usepackage{amsmath}

\begin{document}
\title{矩(moment)}
\author{}
\maketitle


\paragraph{在物理学中}
矩表示距离$r$和物理量$v$的乘积($r\times v$),表征物体的空间分布.

比如\emph{力矩},物体上某点的力矩为$\mu=r^nQ$,其中$r$为到某参考点的距离,$n$代表矩的阶数,如$n=0$为零阶矩,$n=1$为一阶矩等.$Q$为该点上的力.物体上多个点上的力矩则涉及到积分$\mu=\int{r^nQ(r)dr}$.

如果其中物理量是质量$m$,

零阶矩为总质量$\int{m_rdr}$,

一阶矩为重心$\int{rm_rdr}$,

二阶矩为转动惯量$\int{r^2m_rdr}$.

\paragraph{在统计学中}
矩可以用来表征变量的分布.

比如\emph{期望}(一阶原点矩)$E=\int{xp(x)dx}$.

比如\emph{方差}(二阶中心矩)$Var=\int{(x-Ex)^2p(x)dx}$.

比如偏态(三阶中心矩)$S=\int{(x-Ex)^3f(x)dx}$
\paragraph{在图像处理中}
矩可以用来描述图像的某些特征.

零阶原点矩:$M_{00}=\int{\int_{D}{x^0y^0f(x,y)dxdy}}$,可以用于描述光斑的面积(对于二值图像).

一阶原点矩:$M_{10}=\int{\int_{D}{x^1y^0f(x,y)dxdy}}$,$M_{01}=\int{\int{x^0y^1f(x,y)dxdy}}$, 当其除以$M_{00}$ 可用于描述光斑质心$(\frac{M_{10}}{M_{00}},\frac{M_{01}}{M_{00}})$.

二阶中心矩:二阶中心矩一般需要组合成二阶矩阵.
$$\left[
    \begin{matrix}
    M_{20} & M_{11} \\
    M_{11} & M_{02}
    \end{matrix}
\right] \eqno{(1)}$$
借用统计学中二阶中心矩的理解,可以将$M_{20}$理解为在$x$方向上的方差(波动程度),将$M_{02}$理解为在$y$ 方向上的方差.将$M_{11}$理解为光斑在$x$方向和$y$方向上的协方差.也即将上述二阶矩阵理解为光斑的\emph{协方差矩阵}.

事实上,式$(1)$的二阶矩阵描述了二维随机变量(光斑)的分布情况,一般情况下可以用一个椭圆去拟合这个分布.

\subparagraph{椭圆的求解}
式$(1)$描述的椭圆不是椭圆的标准公式.\footnote{注:此时$M_{20}=\sum{(x-x_c)^2I(x,y)}$,而$I(x,y)$为二值化的图像.}
$$
\left[
    \begin{matrix}
        x \\
        y
    \end{matrix}
\right]
\left[
    \begin{matrix}
    M_{20} & M_{11} \\
    M_{11} & M_{02}
    \end{matrix}
\right]
\left[
    \begin{matrix}
        x & y
    \end{matrix}
\right]=0$$

需要通过二次型化简\footnote{线性代数},化成标准的椭圆公式(标准的二次型).

令$M=\left[
    \begin{matrix}
    M_{20} & M_{11} \\
    M_{11} & M_{02}
    \end{matrix}
\right]$,先求其特征值和特征向量.

\begin{align}
|M-\lambda E|&=
\left|
    \begin{matrix}
    M_{20}-\lambda & M_{11} \\
    M_{11} & M_{02}-\lambda
    \end{matrix}
\right| \notag \\
&=(M_{20}-\lambda)(M_{02}-\lambda)-M_{11}^2 \notag\\
&=\lambda^2-(M_{20}+M_{02})\lambda+M_{20}M_{02}-M_{11}^2 \notag
\end{align}
所以:
$$\lambda=\frac{(M_{20}+M_{02})\pm \sqrt{(M_{20}+M_{02})^2-4(M_{20}M_{02}-M_{11}^2)}}{2}$$
经过化简:
$$\lambda_1=\frac{(M_{20}+M_{02})+ \sqrt{(M_{20}-M_{02})^2+4M_{11}^2}}{2}$$
$$\lambda_2=\frac{(M_{20}+M_{02})- \sqrt{(M_{20}-M_{02})^2+4M_{11}^2}}{2}$$


特征向量这里省略,至此求得式(1)标准型为:
$$\left[
    \begin{matrix}
    \frac{(M_{20}+M_{02})+ \sqrt{(M_{20}-M_{02})^2+4M_{11}^2}}{2} & 0 \\
    0 & \frac{(M_{20}+M_{02})- \sqrt{(M_{20}-M_{02})^2+4M_{11}^2}}{2}
    \end{matrix}
\right]$$
根据椭圆标准公式\footnote{$Ax^2+By^2+C=0$},可以看出椭圆两轴分别为
$$l=\sqrt{\frac{\lambda_1}{M_{00}}}$$
$$w=\sqrt{\frac{\lambda_2}{M_{00}}}$$	


\paragraph{关于椭圆}
椭圆是在笛卡尔坐标系\footnote{直角坐标系}上如下形式的方程所定义的曲线.
$$Ax^2+Bxy+Cy^2+Dx+Ey+F=0$$
其中$B^2<4AC$且所有系数均为实数,存在定义在椭圆上的点$(x,y)$的多于一个的解.

中心为于原点的椭圆方程为
$$Ax^2+Bxy+Cy^2=1$$



	
\end{document}