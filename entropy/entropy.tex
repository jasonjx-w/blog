\documentclass{ctexart}
\usepackage{graphicx}
\usepackage[colorlinks,linkcolor=black,anchorcolor=black,citecolor=black]{hyperref}

\begin{document}
\title{熵(entropy)}
\maketitle

\paragraph{信息量}
信息的衡量,也可以理解为不确定度.当某件事的不确定度很高时,其包含的信息量也越大.

如果$x$是0到100之间的某个数字,采用二分法猜的话,需要$log_2{100}$次猜测才能保证猜中.因此0到100中的某个数字$x$的信息量是$log_2{100}$.如果改为0到1000,不确定增加的话,信息量也从$log_2{100}$增加至$log_2{1000}$.

如果从每个数字等于$x$的概率为$\frac{1}{100}$的角度考虑,上述例子的信息量可以表达为$-log_2{\frac{1}{100}}$,和$-log_2{\frac{1}{1000}}$.

这里考虑每个数字等于$x$的概率不相等的情况.将$x$是0到100之间的某数字,改为$x$的可能取值是A,B,C,D中的某个字母,且其对应概率为$\frac{1}{2}$,$\frac{1}{4}$,$\frac{1}{8}$,$\frac{1}{8}$.
ABCD的信息量分别为$-log_2{\frac{1}{2}}$,$-log_2{\frac{1}{4}}$,$-log_2{\frac{1}{8}}$,$-log_2{\frac{1}{8}}$.

\paragraph{信息熵}用来衡量平均而言某个事件的信息量,也就是某事件信息量的期望.上述的例子的熵分别为$\sum_0^{100} \frac{1}{100}\log_2{100}$和$-(\frac{1}{2}\log_2{\frac{1}{2}}+\frac{1}{4}\log_2{\frac{1}{4}}+\frac{1}{8}\log_2{\frac{1}{8}}+\frac{1}{8}\log_2{\frac{1}{8}})$.
所以信息熵的定义为:$$H(X)=-\sum P(x)\log{P(x)}$$

\paragraph{最大熵原理} 当面对未知情况时,应当对其概率分布作最通用的假设,而不假设其为某种特殊情况\footnote{特殊情况的信息熵较小,而一般情况的信息熵较大(一般情况指各个事件出现概率均等).}.比如当预测一个骰子掷出的点数时,应当假设各个数字出现的概率一致($\frac{1}{6}$),而非假设某个数字出现的概率是一个特殊的概率(如$\frac{1}{3}$,而其他数字概率为$\frac{2}{15}$).

此时,概率分布最为均匀,而预测风险最小,并且此时信息熵最大,所以这种模型称为“最大熵模型”\cite{wujun-2008}.	







\bibliographystyle{plain}
\bibliography{refs.bib}
	
\end{document}